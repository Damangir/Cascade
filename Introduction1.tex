\section{Introduction} \label{intro}


White matter changes (WMC) are radiological findings, which are classically defined as areas with relatively high signal intensities on T2-weighted images and low intensities on T1-weighted images. The presence and spatial patterns of WMC on MRI and the appearance of these changes are possible key components for studying this pathology and for prospective clinical practice (diagnosis, follow progression, monitor treatments).

Although these changes are visually appreciable there is no concrete definition and no ground truth is available, which makes segmentation and measurement a daunting task. Although many automatic methods have been proposed in the last 15 years, no single method is widely employed \cite{GarciaReview}. Table \ref{Table:Requirements} presents the desirable characteristics of an algorithm to be widely used.

\begin{table} 
    \begin{tabular}{ c }
        No manual editing \\ 
        Input – any of conventional MRI sequences \\ 
        Results independent of scanning acquisition parameters \\ 
        Should handle diffuse dirty white matter \\ 
        Should handle partial volumes \\ 
        Usable for multi-center datasets \\ 
    \end{tabular} 
    \caption{Desirable features for a white matter change segmentation algorithm } 
    \label{Table:Requirements}
\end{table}
    
Recently, we have developed a fully automatic WMC (age-related WMC, MS lesion) segmentation algorithm \cite{CascadeOrig} that employed multispectral MRI information and a supervised machine learning method. It depended on manual delineation of the WMC, produced sharp segmentation boundaries without taking partial volumes into account and without handling dirty white matter.
    
he aim of this study is to create a robust method having as many characteristics of an ideal algorithm as possible. The method is based on a proposed strict statistical definition of WMC in order to achieve the desirable features. The statistical definition is inherently reproducible and free from manual training; and is applicable to several widely used MRI sequences (T1-weighted, T2-weighted, FLAIR and PD).

The method was tested given any combinations of available sequences (T1-weighted, T2-weighted, FLAIR and PD) and validated against manual delineation, which is considered as the golden standard. The results of different combinations are also cross-validated against one another to assess the expected robustness of the method in order to assure usability for analyzing multi-center datasets.

  
  