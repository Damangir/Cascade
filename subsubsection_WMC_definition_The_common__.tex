\subsubsection{WMC definition}
The common definition of WMC is based on their visual properties on specific sequences (hyper- or hypo-intensities) which has been proved to not to be reproducible enough for large multi-center studies. 
\par
We incorporated the common definition of WMC in a definition as a concrete statistical one which can be robustly measured. Here we  define WMC as areas where their local image histogram are significantly different from the expected normal local histogram on one-tail test.
\par
It is important to recognize that our statistical definition differs from learning methods and outlier detection method, where statistical features of manual delineated WMC is captured in either supervised or unsupervised way.
On the other hand we define WMC and show segmentation based on it corresponds to those based on current accepted definition.
\par
The proposed definition depends on calculating expected normal local histogram. In the present paper expected local histograms of normal brains are calculated for each voxel as the average of local histogram of evidently normal voxels in the same subject. Evidently normal voxels are calculated in two thresholding steps:
\begin{enumerate}
\item Heuristic thresholding is used to capture the bottom percentile (hypo-intense area) of voxels for FLAIR (45\%), T2 (50\%) and PD (65\%) and the upper percentile (hyper-intense area) of voxels for T1 (15\%). The thresholding performed in 1mm, 2mm and 3mm spatial scales. This step should generally be expected to capture the majority of non-WMC voxels (results in step 1 in figure 2).
\item Calculated thresholding is calculated as the optimal threshold \cite{ReduceSVM} to replicate the masks created for all mask at all steps in the previous step.
\end{enumerate}

Proper closing morphological filter is then used to include small holes and missing voxels in the evidently normal brain. Proper opening is defined as $ Min(mask,O( C( O( mask )))  ) $ where $ C $ and $ O $ are morphological closing and opening with two-millimeter spherical structuring element (results in step 2 in figure 2).

    
    
    
    
    
    
    
    
    
  
  
  
  
  