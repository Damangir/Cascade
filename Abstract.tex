We present a method, based on a proposed statistical definition of white matter changes (WMC), which can work with any combination of conventional magnetic resonance (MR) sequences and does not depend on manually delineated samples. T1-weighted, T2-weighted, FLAIR and PD sequences acquired at 1.5 Tesla from 119 subjects from the Kings Health Partners-Dementia Case Register (healthy controls, mild cognitive impairment, Alzheimer’s disease) were used. The presented method has been tested, given all possible combinations of input sequences, against manual segmentations. Strong correlations (r2 = 0.9-0.99) for the volumetric results and high similarity (Dice = 0.85-0.91) were observed when were compared to the manual measurements. Cross-testing of different combinations of input sequences to one another yielded strong correlations (r2 = 0.9-0.99) as well as high similarity (Dice = 0.83-0.94) which exceed intra-rater similarity (Dice = 0.75-0.91). This suggests the possibility of using the proposed method in multi-center studies with different imaging protocols.
Overall, the presented definition is shown to produce accurate results that are more reproducible than manual delineation.

Keywords: segmentation, white matter changes, white matter lesion, cerebrovascular diseases, multiple sclerosis
