We present a method, based on a proposed statistical definition of white matter changes (WMC), which can work with any combination of conventional magnetic resonance (MR) sequences and does not depend on manually delineated samples. T1-weighted, T2-weighted, FLAIR and PD sequences acquired at 1.5 Tesla from 119 subjects from the Kings Health Partners-Dementia Case Register (healthy controls, mild cognitive impairment, Alzheimer’s disease) were used to assess the quality of the proposed definition from three different aspects:
\\*
First, the presented method has been tested, given all possible combinations of input sequences, against manual segmentations. Strong correlations (r\textsuperscript{2} = 0.9-0.99) for the volumetric results and high similarity (Dice = 0.85-0.91) were observed when were compared to the manual measurements; which proves its usability as replacement of manual assessment.
Second, it has been compared with two other state-of-the-art methods and shown it has slightly higher similarity to manual measurement (Dice = 0.90 \textit{vs.} 0.87 and 0.89) compared to them.
Third, cross-testing of different combinations of input sequences to one another performed to simulate the use in multi-center studies with different imaging protocol. It yielded strong correlations (r\textsuperscript{2} = 0.9-0.99) as well as high similarity (Dice = 0.83-0.94) that exceeds manual intra-rater similarity (Dice = 0.75-0.91); which proves it is robust enough for comparing results from different imaging protocols.
\\*
Overall, the presented definition is shown to produce accurate WMC measurement that are more reproducible than manual delineation.

\textbf{Keywords:} segmentation, definition, white matter changes, white matter lesion, cerebrovascular diseases

  
  
  
  
  
  
  
  
  
  
  