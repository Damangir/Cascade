\begin{table}
\centering
    \begin{tabular}  {  r | c c c c c c c }
         & \cite{Zijdenbos_2002} & \cite{LesionTOADS_2010} & \cite{OutlierLocalIntensity_2011} & \cite{CascadeOrig} & \cite{LesionSegmentationToolbox_2012} & \cite{kNN-TTPs_2013} & \cite{Rotation-invariant_2015} \\
         & ANN  & Clustering  & OD  & SVM  & OD and RG  & kNN & SSS \\
    \hline %  
        No manual editing                                      & yes & yes & yes & yes & yes & yes & yes \\ 
        Input – any of conventional MRI sequences              & yes & yes & yes & yes & yes & yes & yes \\ 
        Results independent of scanning acquisition parameters & yes & yes & yes & yes & yes & yes & yes \\ 
        Should handle diffuse dirty white matter               & yes & yes & yes & yes & yes & yes & yes \\ 
        Should handle partial volumes                          & yes & yes & yes & yes & yes & yes & yes \\ 
        Usable for multi-center datasets                       & yes & yes & yes & yes & yes & yes & yes \\ 
    \hline
    \end{tabular} 
    \caption{Desirable features for a white matter change segmentation algorithm and their availability in different methods. ANN: Arteficial neural network, OD: Outlier detection, SVM: Support vector machines, RG: Region growing, kNN: k-Nearest Neighbors algorithm} 
    \label{Table:Other_Methods}
\end{table}
  
  
  
  
  
  
  
  