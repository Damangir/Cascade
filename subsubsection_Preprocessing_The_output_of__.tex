\subsubsection{Preprocessing}
The output of preprocessing is co-registered and inhomogeneity corrected images and brain tissues segmentation. The following enumerates preprocessing steps:
\begin{enumerate}
\item Intra-subject registration (reference image is user’s choice) using rigid 3D transformation with mutual information
\item Standard space to subject registration using affine 3D transformation between the native and MNI space using mutual information
\item Skull stripping: MNI brain mask is registered to native space. The registered brain mask is eroded with a 5mm spherical structuring element and the eroded mask is extended using a region-growing algorithm.
\item Inhomogeneity correction performed on all registered input images using N3 algorithm \cite{N3}
\item Brain tissue segmentation using Bayesian inference
\begin{itemize}
    \item Prior probability distribution is estimated by registering the average MNI tissue maps to the native space. Tissue model is the marginal probability distribution of intensities of the 3 tissue types, which is calculated by using the algorithm described in \cite{EM_Empirical}. Initial brain tissue segmentation is performed using the tissue model and prior probabilities
    \item Updating the initial tissue segmentation using a new tissue probability model at each voxel which is the relative frequency of each tissue type in adjacent voxels in a sphere of radius 1mm
    \item Removing potential WMC that was classified as GM: GM voxels that are bright on T2 images (>85\% of GM) or FLAIR image (>90\% of GM) are labeled as suspect voxels. Suspected voxels surrounded by mostly WM will be labeled as WM and voxels surrounded by mostly GM will be labeled back to GM
\end{itemize}
\end{enumerate}
  
  
  
  
  
  
  