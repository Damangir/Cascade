\subsubsection{Preprocessing}
The output of preprocessing is co-registered and inhomogeneity corrected images and brain tissues segmentation. The following enumerates preprocessing steps:
\begin{enumerate}
\item Intra-subject registration (reference image is user’s choice) using rigid 3D transformation with mutual information
\item Skull stripping: Affine transformation between the native and MNI space is calculated using mutual information and the MNI brain mask is registered to native space. The registered brain mask is eroded with a 5mm spherical structuring element and the eroded mask is extended using a region-growing algorithm
\item Inhomogeneity correction performed on all registered input images using N3 algorithm (Sled, Zijdenbos, and Evans 1998)
\item Brain tissue segmentation using Bayesian inference
    \begin{enumerate}
    \item Spline-based nonlinear transformation between native and MNI space is calculated considering only voxels in the brain mask and using mutual information as before. Prior probability distribution is estimated by registering the average MNI tissue maps to the native space using the nonlinear transform. Tissue model is the marginal probability distribution of intensities of the 3 tissue types, which is calculated by using the algorithm described in (El-Baz and Gimel’farb 2007). Initial brain tissue segmentation is performed using the tissue model and prior probabilities.
    \item Smoothing by calculating a new tissue probability model at each voxel, that calculates the relative frequency of each tissue type in adjacent voxels in a sphere of radius 1mm.
    \item Relabeling WMC that was classified as GM. GM voxels that are bright on T2 images (intensity >85 percentile of GM) are labeled as suspect voxels, which are relabeled according to the tissue type of the majority of the neighboring voxels in an iterative process. Voxels surrounded by mostly WM will be labeled as WM and voxels surrounded by mostly GM will be labeled back to GM.
    \end{enumerate}
\end{enumerate}