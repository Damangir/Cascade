\subsection{Subjects}
MRI scans of 119 subjects (Alzheimer’s disease, mild cognitive impairment (MCI) and healthy controls) were used from Kings Health Partners-Dementia Case Register (KHP-DCR) in UK. The AD diagnosis was made according to Diagnostic and Statistical Manual for Mental Diagnosis, fourth edition and MCI was defined according to Petersen criteria (Petersen et al. 1999). Subjects were 76.4 (7.4) years old, 56% female, had 12.0 (4.3) years education and mini-mental state examination (MMSE) 26.5 (4.8). 
The imaging protocol included the following sequences: sagittal 3D T1-weighted MPRAGE (1.1×1.1×1.2 mm3), axial proton density (PD), T2-weighted fast spin echo image and FLAIR. All images had full brain and skull coverage and quality control was performed according to the AddNeuroMed procedure (Simmons et al. 2011). In the rest of the paper, T1 refers to T1-weighted MPRAGE and T2 refers to T2-weighted image.

    