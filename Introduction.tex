\section{Introduction} \label{intro}


White matter changes (WMC) are radiological findings, which are classically defined as areas with relatively high signal intensities on T2-weighted images (T2) and low intensities on T1-weighted images (T1). The presence and spatial patterns of WMC on MRI and the appearance of these changes are possible key components for studying this pathology and for prospective clinical practice (diagnosis, follow progression, monitor treatments).

Although these changes are visually appreciable there is neither a concrete definition nor a ground truth is available, which makes segmentation and measurement a daunting task. Although many automatic methods have been proposed in the last 15 years \cite{Kamber_1995,Udupa_1997,Boudraa2000,Zijdenbos_2002,LesionTOADS_2010,OutlierLocalIntensity_2011,CascadeOrig,LesionSegmentationToolbox_2012,kNN-TTPs_2013,Rotation-invariant_2015}; no single method is widely employed, nor satisfies all desirable characteristics of being widely used \cite{GarciaReview}. Table \ref{Table:Requirements} presents the desirable characteristics of an algorithm to be widely used.

\begin{table}
\centering
    \begin{tabular}  {  c }
    \hline
        No manual editing \\ 
        Input – any of conventional MRI sequences \\ 
        Results independent of scanning acquisition parameters \\ 
        Should handle diffuse dirty white matter \\ 
        Should handle partial volumes \\ 
        Usable for multi-center datasets \\
    \hline
    \end{tabular} 
    \caption{Desirable features for a white matter change segmentation algorithm } 
    \label{Table:Requirements}
\end{table}
    


We believe failing to address these specification rooted in the fact that the problem of segmenting WMC as viewed today is an \textit{ill-posed} problem \cite{IllPosed}; meaning it does not have a unique solution. In other words, there is no single WMC segmentation for a given brain scan that everyone agrees on. This characteristic is unique to WMC segmentation when compared to other segmentation questions like whole brain or similar segmentations; that is why there are widely used and accepted methods for those questions.

To our knowledge, previously presented methods have used machine learning (supervised or unsupervised) in order to produce a segmentation as close to manual WMC segmentation as possible. Since those method are aiming at an undefined and moving target (i.e. the problem is \textit{ill-posed}) they were not be able to be reliable enough. It does not mean that those methods are not producing segmentation close to manual segmentations, rather it suggests that those methods need to be optimized for each particular research. We thus believe that although this approach can be and has been useful in some scenarios, it can not lead to a general solution. On the other hand, if the problem had been \textit{well-posed}, it could have stood a foundation for a stable computer solution.

In the present work, we reformulate the problem of WMC segmentation as a \textit{well-posed} problem and show that the reformulated problem yields the same results as the traditional one. In this way, a concrete statistical definition of WMC is presented and the segmentation is performed based on the statistical definition. In order to assess the validity of the proposed definition, result from proposed statistical definition using all combination input sequences (e.g. FLAIR and T1; T2 and T1; and etc.) have been:

\begin{enumerate}
\item Compared with manual delineation
\item Compared with other state-of-the-art methods
\item Cross-compared to one another (e.g. Results using FLAIR and T1 compared with the one using T2 and T1)
\end{enumerate}

In this paper, a dataset with four widely used MRI sequences (T1, T2, FLAIR and PD) and manual WMC delineation has been used. After describing the dataset, the proposed definition of WMC will be presented followed by step-by-step description of all necessary pre-processing and its implementation. Then the experimental setup and its results are presented before discussing the method and implication of the results.

  
  
  
  
  
  
  
  
  
  
  