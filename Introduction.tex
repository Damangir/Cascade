\section{Introduction} \label{intro}


White matter changes (WMC) are radiological findings, which are classically defined as areas with relatively high signal intensities on T2-weighted images and low intensities on T1-weighted images. The presence and spatial patterns of WMC on MRI and the appearance of these changes are possible key components for studying this pathology and for prospective clinical practice (diagnosis, follow progression, monitor treatments).

Although these changes are visually appreciable there is neither a concrete definition nor a ground truth is available, which makes segmentation and measurement a daunting task. Although many automatic methods have been proposed in the last 15 years; no single method is widely employed, nor satisfies all desirable characteristics of being widely used \cite{GarciaReview}. Table \ref{Table:Requirements} presents the desirable characteristics of an algorithm to be widely used.

\begin{table}
\centering
    \begin{tabular}  {  c }
    \hline
        No manual editing \\ 
        Input – any of conventional MRI sequences \\ 
        Results independent of scanning acquisition parameters \\ 
        Should handle diffuse dirty white matter \\ 
        Should handle partial volumes \\ 
        Usable for multi-center datasets \\
    \hline
    \end{tabular} 
    \caption{Desirable features for a white matter change segmentation algorithm } 
    \label{Table:Requirements}
\end{table}
    
We believe failing to address all these specification rooted in the fact that the problem of segmenting WMC as viewed today is a \textit{ill-posed} problem \cite{IllPosed}; meaning it does not have a unique solution. In other words, there is no single WMC segmentation for a given brain scan that everyone agrees on. This characteristic is unique to WCM segmentation when compared to other segmentation questions like whole brain or similar segmentations; that is why there are widely used and accepted methods for those questions.

To our knowledge, previously presented methods have used machine learning (supervised or unsupervised) in order to produce a segmentation as close to manual WMC segmentation as possible. Since those method are aiming at an undefined and moving target (i.e. the problem is \textit{ill-posed}) they were not be able to be reliable enough. It does not mean that those methods are not producing segmentation close to manual segmentations, rather it suggests that those methods need to be optimized for each particular research. We thus believe that although this approach can be and has been useful in some scenarios, it can not lead to a general solution. On the other hand, if the problem had been \textit{well-posed}, then it could have stood a foundation of a solution on a computer using a stable algorithm.

In the present work, we aim to reformulate the problem of WMC segmentation as a \textit{well-posed} problem and based on that create a robust method having as many characteristics of an ideal algorithm (table \ref{Table:Requirements}) as possible. In this way, a concrete statistical definition of WMC is presented and the segmentation is performed based on the statistical definition. The statistical definition is inherently reproducible and free from manual training; and is applicable to several widely used MRI sequences (T1-weighted, T2-weighted, FLAIR and PD). The approach based on reformulated problem is then assessed in three different ways to show its validity as a replacement to current problem formulation:

\begin{enumerate}
\item It has been tested, given all possible combinations of input sequences, against manual segmentations to proves its usability as replacement of manual assessment
\item Its results have been compared with two other state-of-the-art methods'
\item Result from different input sequences have been cross-tested to one another to compare with traditional intra-rater similarity to suggest it is robustness in multi-center studies with different imaging protocols. 
\end{enumerate}

